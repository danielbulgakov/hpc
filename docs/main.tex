% Compiler parameters

\documentclass[12pt]{extarticle}
\usepackage[utf8]{inputenc}
\usepackage{fontspec}
\defaultfontfeatures{Ligatures={TeX},Renderer=Basic}
\setmainfont[Ligatures={TeX,Historic}]{Times New Roman}
\usepackage[a4paper,
left=25mm,
right=15mm,
top=20mm,
bottom=20mm]{geometry}
\usepackage[linktoc=all]{hyperref}
\usepackage{titlesec}
\titlelabel{\thetitle.\quad}
\usepackage{tocloft}

\renewcommand{\cftsecfont}{}% Remove \bfseries from section titles in ToC
\renewcommand{\cftsecpagefont}{}% Remove \bfseries from section titles' page in ToC
\renewcommand{\cftsecaftersnum}{.}%

\usepackage{titlesec}
\usepackage{todonotes}

\titleformat*{\section}{\large\bfseries} % Здесь можно указать желаемый размер шрифта для \section
\titleformat*{\subsection}{\large\bfseries} % Здесь можно указать желаемый размер шрифта для \subsection
\titleformat*{\subsubsection}{\large\bfseries}

\setlength{\parindent}{1.25cm}
\setlength{\parskip}{0.4cm}
\font\subtitlefont=cmr12 at 12pt
\font\titlefont=cmr12 at 24pt
\usepackage{color}
\usepackage{mathtools}
\usepackage{listings}
\usepackage{graphicx}
\usepackage{tocloft}
\usepackage{indentfirst}
\usepackage{enumitem}
\usepackage{graphicx}
\usepackage{subcaption}
\usepackage{babel}
\usepackage{setspace}
\renewcommand{\contentsname}{}
\renewcommand{\cftsecleader}{\cftdotfill{\cftdotsep}}
\graphicspath{ {./resources/} }

\usepackage{listings}
\definecolor{dkgreen}{rgb}{0,0.6,0}
\definecolor{gray}{rgb}{0.5,0.5,0.5}
\definecolor{mauve}{rgb}{0.58,0,0.82}
\lstset{
    language=Python,
    basicstyle=\ttfamily\small,
    keywordstyle=\color{blue},
    commentstyle=\color{dkgreen},
    stringstyle=\color{mauve},
    stepnumber=1,
    breaklines=true,
    breakatwhitespace=true,
    tabsize=4,
    captionpos=tl,
}

% End of compiler parameters

\title{}
\author{}
\date{}

\begin{document}

    \begin{titlepage}

        \begin{center}
            МИНИСТЕРСТВО ОБРАЗОВАНИЯ И НАУКИ РОССИЙСКОЙ ФЕДЕРАЦИИ \\
            Федеральное государственное автономное образовательное \\
            учреждение высшего образования \\
            «Национальный исследовательский Нижегородский государственный университет им. Н.И. Лобачевского»\\
        \end{center}

        \bigbreak

        \begin{center}
            Институт информационных технологий, математики и механики \\
            {\bfseries Кафедра Математического обеспечения и суперкомпьютерных технологий} \\
        \end{center}

        \vspace{2em}

        \begin{center}
            ОТЧЁТ \\ по предмету \\
            {\bfseries ``Анализ производительности и оптимизация программного
            обеспечения''}
        \end{center}

        \vspace{5em}

        \begin{flushright}
            {\bfseries Выполнил:} \\студент группы 3824М1ПР1\\Булгаков Д.Э.\\
            \hfill Подпись \hspace{5em} \newline \\
            {\bfseries Проверил:} \\к.т.н., доц.\\ И.Б. Мееров\\
            \hfill Подпись \hspace{5em} \newline \\
        \end{flushright}


        \vspace{\fill}

        \begin{center}
            Нижний Новгород\\2024
        \end{center}
    \end{titlepage}

    \begin{spacing}{1.5}

    % Содержание

    \tableofcontents
    \thispagestyle{empty}
    \newpage

    \pagestyle{plain}
    \setcounter{page}{3}

    \newpage
    \section{Введение}

    Анализ производительности и оптимизация программного обеспечения (ПО) являются важными аспектами разработки масштабируемых и эффективных решений. Выявление узких мест, создание эффективного кода и использование современных инструментов способствуют значительному улучшению качества ПО и удовлетворению потребностей конечных пользователей.

    В данном отчете рассматривается задача решения систем линейных алгебраических уравнений (СЛАУ) с использованием метода коньюгированных градиентов. Данная задача имеет высокую востребованность в численных методах и широко используется в различных областях науки и инженерии.

    Цель работы — анализ различных методов оптимизации на основе готового решения из открытого репозитория на GitHub и оценка их влияния на производительность алгоритма.

    \newpage
    \section{Описание алгоритма}

    Метод сопряженных градиентов (Conjugate Gradient Method) — это численный алгоритм, предназначенный для решения систем линейных алгебраических уравнений вида 
    \[
    A \cdot x = b,
    \]
    где \(A\) является симметричной положительно определённой матрицей. Этот метод является итерационным, что делает его эффективным для решения задач с большой размерностью, особенно когда матрица \(A\) разрежена.

    Алгоритм включает следующие основные шаги:

    \subsection{Инициализация}
    Задаются начальное приближение \(x_0\), начальный вектор остатка \(r_0 = b - A \cdot x_0\) и вспомогательный вектор \(p_0 = r_0\).

    \subsection{Итерационный процесс}
    На каждом шаге итерации вычисляются:
    \begin{itemize}
        \item \(\alpha_k = \frac{r_k^T \cdot r_k}{p_k^T \cdot A \cdot p_k}\) — коэффициент для обновления решения;
        \item \(x_{k+1} = x_k + \alpha_k \cdot p_k\) — новое приближение решения;
        \item \(r_{k+1} = r_k - \alpha_k \cdot A \cdot p_k\) — новый вектор остатка;
        \item \(\beta_k = \frac{r_{k+1}^T \cdot r_{k+1}}{r_k^T \cdot r_k}\) — коэффициент для обновления направления поиска;
        \item \(p_{k+1} = r_{k+1} + \beta_k \cdot p_k\) — новое направление поиска.
    \end{itemize}

    \subsection{Проверка сходимости}
    Итерации продолжаются до тех пор, пока норма вектора остатка \(\|r_k\|\) не станет меньше заданного порога (показателя точности).

    \subsection{Преимущества метода}
    Метод сопряженных градиентов обладает следующими преимуществами:
    \begin{itemize}
        \item Экономия памяти, так как он не требует хранения всей матрицы \(A\), а только её умножения на вектор;
        \item Быстрая сходимость для хорошо обусловленных систем.
    \end{itemize}

    Однако для достижения высокой производительности при реализации алгоритма важно учитывать аспекты, связанные с доступом к памяти, балансировкой вычислений и возможностями параллельной обработки. В следующем разделе будут рассмотрены методы ускорения алгоритма, включая применение технологий параллелизма.


    \newpage
    \section{Способы ускорения}

    Для повышения производительности алгоритма решения системы линейных уравнений методом сопряженных градиентов были использованы следующие подходы:

    \subsection{Использование технологии OpenMP}
    Для распараллеливания вычислений в коде применяется технология OpenMP, которая позволяет задействовать многопоточность. Это обеспечивает ускорение выполнения следующих операций:
    \begin{itemize}
        \item Векторно-матричное произведение (\texttt{omp\_matrix\_vec}) распараллеливается по строкам матрицы с использованием директивы \texttt{\#pragma omp parallel for}.
        \item Скалярное произведение векторов (\texttt{omp\_vec\_vec}) выполняется с применением директивы \texttt{\#pragma omp parallel for reduction(+ : res)} для обеспечения корректного суммирования результатов из разных потоков.
    \end{itemize}

    \subsection{Распараллеливание независимых секций}
    Для повышения эффективности в цикле алгоритма сопряженных градиентов используется директива \texttt{\#pragma omp parallel sections}, которая позволяет выполнять независимые вычисления, такие как обновление решения (\(x\)) и остатка (\(r\)), параллельно.

    \subsection{Оптимизация численных операций}
    Для предотвращения деления на ноль и улучшения численной устойчивости введено использование параметра \texttt{SMOL}, минимального значения, ниже которого дробь обнуляется:
    \[
    d = \frac{r^T r}{\max(p^T (A \cdot p), \texttt{SMOL})}.
    \]
    Это позволяет избежать ошибок вычислений при работе с малыми числами.

    \subsection{Эффективная работа с памятью}
    \begin{itemize}
        \item Матрица и векторы передаются по ссылке, что минимизирует копирование данных.
        \item Векторы обновляются на месте, что уменьшает количество выделений и освобождений памяти.
    \end{itemize}

    \newpage
    \section{Результаты}

    Испытания проведены на матрице размера \(1000 \times 1000\) и векторе размера \(1000\).

    \subsection{Intel Advisor}

    \par
    Первым инструментом для сравнения производительности был Intel Advisor. На нем получились следующие результаты:

    \par \textbf{Оригинальная версия}

    \begin{figure}[ht]
        \centering
        \includegraphics[width=\linewidth]{/home/dbulgakov/Study/hpc_screens/no_omp_result.png}
    \end{figure}

    \par \textbf{Оптимизированная версия}

    \begin{figure}[ht]
        \centering
        \includegraphics[width=\linewidth]{/home/dbulgakov/Study/hpc_screens/omp_result.png}
    \end{figure}

    На основе этих данных можно сделать следующие выводы:

    \begin{enumerate}
        \item \textbf{Производительность:}
        \begin{itemize}
            \item Оригинальная версия: 0.363 GFLOPS.
            \item Оптимизированная версия: 4.701 GFLOPS.
        \end{itemize}
        Ускорение составляет примерно \textbf{13 раз}.
    
        \item \textbf{Время выполнения:}
        \begin{itemize}
            \item Оригинальная версия: общее время выполнения \textbf{25.580 секунд}.
            \item Оптимизированная версия: общее время выполнения \textbf{1.790 секунд}.
        \end{itemize}
        Ускорение примерно в \textbf{14 раз}.
    
        \item \textbf{Затраты времени на ключевые операции (Self Time):}
        \begin{itemize}
            \item Оригинальная версия: \textbf{11.051 секунд}.
            \item Оптимизированная версия: \textbf{0.852 секунды}.
        \end{itemize}
        Значительное сокращение времени выполнения.
    
        \item \textbf{Объём операций с памятью (Memory Traffic):}
        \begin{itemize}
            \item Оригинальная версия: общий трафик памяти \textbf{801.537 GB}.
            \item Оптимизированная версия: общий трафик памяти \textbf{68.073 GB}.
        \end{itemize}
        Оптимизация позволила уменьшить объём операций с памятью приблизительно в \textbf{11 раз}.
    \end{enumerate}

    \subsection{AMDuProf}
    Дополнительным инструментом был использован AMDuProf

    \par \textbf{Оригинальная версия}

    \begin{figure}[ht]
        \centering
        \includegraphics[width=\linewidth]{/home/dbulgakov/Study/hpc_screens/amd_no_omp.png}
    \end{figure}

    \begin{figure}[ht]
        \centering
        \includegraphics[width=\linewidth]{/home/dbulgakov/Study/hpc_screens/amd_no_omp_1.png}
    \end{figure}

    \par \textbf{Оптимизированная версия}

    \begin{figure}[ht]
        \centering
        \includegraphics[width=\linewidth]{/home/dbulgakov/Study/hpc_screens/amd_omp.png}
    \end{figure}

    \begin{figure}[ht]
        \centering
        \includegraphics[width=\linewidth]{/home/dbulgakov/Study/hpc_screens/amd_omp_1.png}
    \end{figure}

    На основе метрик и графиков собранных с данном профилировщика, можно сделать следующие выводы:

    \textbf{Оригинальная версия}
    \begin{itemize}
        \item Общее количество активных потоков: 1 (Thread Count = 1). Все вычисления выполняются последовательно.
        \item Время выполнения составляет 6.6385 секунд, что отражает работу одного потока без использования параллелизма.
    \end{itemize}

    \textbf{Оптимизированная версия}
    \begin{itemize}
        \item Общее количество потоков, участвующих в вычислениях, увеличилось до 12.
        \item Общее время выполнения алгоритма существенно снизилось: главный поток выполняет работу за 2.9801 секунды (из общего времени), что демонстрирует успешное использование параллелизма.
    \end{itemize}

    \newpage
    \section{Заключение}

    В данной работе был рассмотрен метод сопряженных градиентов для решения систем линейных уравнений, а также проведён анализ его производительности и реализация с использованием различных подходов к оптимизации. Основное внимание было уделено использованию технологии OpenMP для распараллеливания вычислений, что позволило эффективно задействовать ресурсы многопроцессорных систем. 

    Были выделены ключевые аспекты оптимизации:
    \begin{itemize}
        \item Распараллеливание операций векторно-матричного умножения и скалярного произведения.
        \item Разделение независимых вычислительных секций для ускорения итеративного процесса.
        \item Предотвращение ошибок численной стабильности с использованием параметра \texttt{SMOL}.
        \item Уменьшение накладных расходов на выделение памяти за счёт обновления данных на месте.
    \end{itemize}

    Таким образом, удалось подтвердить значительное ускорение работы параллельной версии по сравнению с последовательной. Полученные результаты демонстрируют эффективность подходов, использованных в реализации. 

    \newpage
    \section{Приложение}

    \subsection{Оригинальный метод}
    \subsubsection*{slau\_gradient\_orig.hh}
    \lstinputlisting[language=c]{/home/dbulgakov/Study/hpc/src/slau_gradient_orig.hh}
    \subsubsection*{slau\_gradient\_orig.cpp}
    \lstinputlisting[language=c]{/home/dbulgakov/Study/hpc/src/slau_gradient_orig.cpp}
    
    \subsection{Оптимизированный метод}

    \subsubsection*{slau\_gradient.hh}
    \lstinputlisting[language=c]{/home/dbulgakov/Study/hpc/src/slau_gradient.hh}
    \subsubsection*{slau\_gradient.cpp}
    \lstinputlisting[language=c]{/home/dbulgakov/Study/hpc/src/slau_gradient.cpp}


    \end{spacing}
\end{document}